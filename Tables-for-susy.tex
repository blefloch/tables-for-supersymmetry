\documentclass[10pt,letterpaper]{article}
\usepackage{savetrees}
\usepackage{amsfonts,amssymb,mathtools,array,booktabs,multirow,bbm,hyperref}
\usepackage{latexsym}
%%% Sorry the next two lines are a bit horrible
\renewcommand{\section}[1]{\stepcounter{section}\medskip\pagebreak[1]\noindent\textbf{\large\arabic{section}.\ #1}\smallskip}
\renewcommand{\paragraph}[1]{\textbf{#1}}

\begin{document}
\twocolumn\raggedbottom

\noindent\textbf{\Large Tables for supersymmetry.} Based on [1-2].

\noindent ``Ed.'': Bruno Le Floch, Princeton University, \today.

\section{Lie (super)algebras}

\paragraph{Simple complex Lie algebras.}
Infinite series $A_{n\geq 1}$, $B_{n\geq 1}$, $C_{n\geq 1}$, $D_{n\geq 2}$ with
$A_1=B_1=C_1$, $B_2=C_2$, $D_2=A_1\oplus A_1$, $D_3=A_3$.
Five exceptions with $\dim(E_6)=78$, $\dim(E_7)=133$, $\dim(E_8)=248$, $\dim(F_4)=52$, $\dim(G_2)=14$.
\begin{center}
\vspace{-.3\baselineskip}
\begin{tabular}{*{3}{>{$}l<{$}}l}\toprule
  \text{Type} & \text{Dimension} & \text{Lie algebra} \\\midrule 
  A_n & n(n+2) & sl(n+1,\mathbb{C}) =\{\text{traceless}\} \\
  B_n & n(2n+1) & so(2n+1,\mathbb{C}) =\{\text{antisymmetric}\} \\
  C_n & n(2n+1) & sp(2n,\mathbb{C})
	=\left\{\left(\begin{smallmatrix}0&\mathbbm{1}_n\\-\mathbbm{1}_n&0\end{smallmatrix}\right)\times\text{symmetric}\right\}
  \\
  D_n & n(2n-1) & so(2n,\mathbb{C}) =\{\text{antisymmetric}\} \\
  \bottomrule
\end{tabular}
\end{center}

\paragraph{Real forms.}  Denote
$sl(n)=sl(n,\mathbb{R})$,
$sp(2n)=sp(2n,\mathbb{R})$,
$su^*(2n)=sl(n,\mathbb{H})$,
$so^*(2n)=o(n,\mathbb{H})$,
$usp(2m,2n)=u(m,n,\mathbb{H})$.
A Lie algebra is ``compact'' if it exponentiates to a compact Lie group.
In $E_{r(s)}$, $s$ is the number of $(\text{non-compact})-(\text{compact})$ generators.
\begin{center}
\vspace{-.3\baselineskip}
\begin{tabular}{@{}c@{ }lll}\toprule
& Real form & \hspace{-1em}Max ``compact'' subalgebra & Range \\
\midrule
\multirow{4}{*}{\rotatebox{90}{$sl(n)$}}
& $su(n)$ & ``compact'' & \\
& $sl(n)$ & $so(n)$ & \\
& $su(n-p,p)$ & $su(n-p)\oplus su(p)\oplus u(1)$ & $0<p<n$ \\
& $su^*(n)$ & $usp(n)$ & $n$ even \\
\midrule
\multirow{3}{*}{\rotatebox{90}{$so(n)$}}
& $so(n)$& ``compact'' & \\
& $so(p,n-p)$& $so(p)\oplus so(n-p)$ & $0<p<n$ \\
& $so^*(2n)$   & $u(n)$ & $n$ even \\
\midrule
\multirow{3}{*}{\rotatebox{90}{$sp(2n)$}}
& $usp(2n)$ & ``compact'' & \\
& $sp(2n)$  & $u(n)$ & \\
& $usp(2n-2p,2p)$ & $usp(2n-2p)\oplus usp(2p)$ & $0<p<n$ \\
\midrule
\multicolumn{4}{c}{%
  \begin{tabular}[c]{ll}
  $E_{6(-78)}$ & ``compact'' \\
  $E_{6(-26)}$ & $F_4$ \\
  $E_{6(-14)}$ & $so(10)\oplus so(2)$\\
  $E_{6(2)}$ & $su(6)\oplus su(2)$\\
  $E_{6(6)}$ & $usp(8)$\\
  \midrule
  $E_{7(-133)}$& ``compact'' \\
  $E_{7(-25)}$& $E_{6,-78}\oplus so(2)$ \\
  $E_{7(-5)}$& $so(12)\oplus su(2)$ \\
  $E_{7(7)}$& $su(8)$
  \end{tabular}\quad
  \begin{tabular}[c]{ll}
  $E_{8(-248)}$& ``compact''\\
  $E_{8(-24)}$&$E_{7,-133}\oplus su(2)$\\
  $E_{8(8)}$&$so(16)$\\
  \midrule
  $G_{2(-14)}$ & ``compact'' \\
  $G_{2(2)}$ & $su(2)\oplus su(2)$ \\
  \midrule
  $F_{4(-52)}$ & ``compact'' \\
  $F_{4(-20)}$ & $so(9)$ \\
  $F_{4(4)}$ & $usp(6)\oplus su(2)$ \\
  \end{tabular}
}\\\bottomrule
\end{tabular}
\end{center}

\paragraph{Accidental isomorphisms.}
\begin{center}
\vspace{-1.5\baselineskip}
\begin{minipage}[t]{.55\linewidth}
\begin{align*}
so(2)&= u(1), \quad so(1,1)=\mathbb{R}\\
so(3)&= su(2)=su^*(2)\\
so(2,1) &=su(1,1)=sl(2)=sp(2)\\
so(4)&=su(2)\oplus su(2)\\
so(3,1)&=sl(2,\mathbb{C})=sp(2,\mathbb{C})\\
so(2,2)&=sl(2)\oplus sl(2)\\
so^*(4)&=su(1,1)\oplus su(2)\\
so(5)&=usp(4)
\end{align*}
\end{minipage}%
\begin{minipage}[t]{.45\linewidth}
\begin{align*}
so(4,1)&=usp(2,2)\\
so(3,2)&=sp(4)\\
so(6)&=su(4)\\
so(5,1)&=su^*(4)\\
so(4,2)&=su(2,2)\\
so(3,3)&=sl(4)\\
so^*(6)&=su(3,1)\\
so^*(8)&=so(6,2)
\end{align*}
\end{minipage}
\end{center}

\paragraph{Classical Lie superalgebras:}
the bosonic algebra acts on the fermionic generators in a completely reducible representation.
This excludes Cartan-type superalgebras $W(n)$, $S(n)$, $\tilde S(n)$ and $H(n)$.
In this table, $m,n\geq 1$ and we do not list purely bosonic Lie algebras.
The factor $\mathbb{C}$ of $sl(m|n)$ must be removed if $m=n$.
\begin{center}
\vspace{-.5\baselineskip}
\begin{tabular}{lll}\toprule
& Bosonic algebra & Fermionic repr. \\\midrule
$sl(m|n)$ & $sl(m,\mathbb{C})\oplus sl(n,\mathbb{C})\oplus\mathbb{C}$ & $(m,\overline{n})\oplus(\overline{m},n)$ \\
$osp(m|2n)$ & $so(m,\mathbb{C}) \oplus sp(2n)$ & $(m,2n)$ \\
$D(2,1,\alpha)$ & $sl(2,\mathbb{C})^3$ & $(2,2,2)$ \\
$F(4)$ & $so(7,\mathbb{C})\oplus sl(2,\mathbb{C})$ & $(8,2)$ \\
$G(3)$ & $G_2\oplus sl(2,\mathbb{C})$ & $(7,2)$ \\
$P(m)$ & $sl(m+1,\mathbb{C})$ & $\text{sym}\oplus(\text{antisym})^*$ \\
$Q(m)$ & $sl(m+1,\mathbb{C})$ & adjoint\\
\bottomrule
\end{tabular}
\end{center}

\paragraph{Real forms of Lie superalgebras,}
starting from their ``compact'' form ($p=q=0$).  $P(m)$ has no ``compact'' form.
Here, $m,n\geq 1$, $0\leq p\leq m/2$, $0\leq q\leq n/2$.
The forms $su^*$, $osp^*$, $Q^*$ only exist for even rank; $sl'$ only if $m=n$.
\begin{center}
\vspace{-.5\baselineskip}
\begin{tabular}{*{2}{>{$}l<{$}}}\toprule
\text{Real form} & \text{Bosonic algebra}  \\ \midrule
su(m-p,p|n-q,q) & su(m-p,p)\oplus su(n-q,q)\oplus u(1)^{\mathsection}\\
sl(m|n) & sl(m)\oplus sl(n)\oplus so(1,1)^{\mathsection} \\
sl'(n|n) \quad\, (m=n)& sl(n,\mathbb{C})\\
su^*(m|n) \:\: (m,n \text{ even}) & su^*(m)\oplus su^*(n)\oplus so(1,1)^{\mathsection}\\
\midrule
osp(m-p,p|2n) & so(m-p,p)\oplus sp(2n) \\
\multicolumn{2}{l}{$osp^*(m|2n-2q,2q)$ ($m$ even)\quad $so^*(m)\oplus usp(2n-2q,2q)$} \\
\midrule
D^p(2,1,\alpha) \;^{\mathparagraph} & so(4-p,p)\oplus sl(2)\quad (p=0,1,2)\\
\midrule
F^p(4) \text{ for $p=0,3$} & so(7-p,p)\oplus sl(2) \\
F^p(4) \text{ for $p=1,2$} & so(7-p,p)\oplus su(2) \\
\midrule
G_s(3) \text{ for $s=-14,2$} & G_{2(s)}\oplus sl(2) \\
\midrule
P(m) & sl(m+1) \\
\midrule
UQ(m-p,p) & su(m+1-p,p) \\
Q(m) & sl(m+1) \\
Q^*(m) \quad (m \text{ odd}) & su^*(m+1) \\
\bottomrule
\end{tabular}
\vspace{-.7\baselineskip}
\end{center}

$^{\mathsection}$
For $m=n$, $u(1)$ and $so(1,1)$ factors are absent.
Additionally, one can project down to a single bosonic factor.

$^{\mathparagraph}$
The three $sl(2)$ bosonic factors of $D(2,1,\alpha)$ appear with weights $1$, $\alpha$ and $-1-\alpha$ in fermion anticommutators.
For $D^0$ and $D^2$, $\alpha$ is real.  For $D^1$, $\alpha=1+ia$ with $a$ real.

\smallskip

\paragraph{Some isomorphisms:}
$su(1,1|1)=sl(2|1)=osp(2|2)$
and $su(2|1)=osp(2^*|2,0)$.  For $\alpha=1$, $D^p(2,1,1)=osp(4-p,p|2)$.

\section{Spinors}

\paragraph{Clifford algebra.}  Let $h_{ab}$ be diagonal with $s$~`$+1$' and $t$~`$-1$', and $d=s+t$.  The Clifford algebra $\{\Gamma_a,\Gamma_b\}=2h_{ab}$ has real dimension~$2^d$ and is isomorphic to a matrix algebra $M_{2^{\#}}(\bullet)$ with
\begin{center}
\vspace{-.5\baselineskip}
\begin{tabular}{>{$}r<{$}*{8}{>{$}c<{$}}}\toprule
s-t \bmod{8} & 0 & 1 & 2 & 3 & 4 & 5 & 6 & 7 \\
\bullet \text{\quad is} & \mathbb{R}& \mathbb{R}\oplus\mathbb{R} & \mathbb{R} & \mathbb{C} & \mathbb{H} & \mathbb{H}\oplus\mathbb{H} & \mathbb{H} &\mathbb{C}\\\bottomrule
\end{tabular}
\end{center}
%with $\bullet = [\mathbb{R}, \mathbb{R}\oplus\mathbb{R}, \mathbb{R}, \mathbb{C}, \mathbb{H}, \mathbb{H}\oplus\mathbb{H}, \mathbb{H}, \mathbb{C}][s-t\bmod{8}]$.


\paragraph{Charge conjugation.}  $(-\eta)\Gamma_a^T=\mathcal{C}\Gamma_a\mathcal{C}^{-1}$ are conjugate for $\eta=\pm 1$ because they obey the same algebra.  Get $\mathcal{C}^T=-\varepsilon \mathcal{C}$ with $\varepsilon=\pm 1$ by transposing twice.
Let $\Gamma^{(n)}=\Gamma_{a_1\ldots a_n}$.
Using $\left(\mathcal{C}\Gamma^{(n)}\right) ^T= -\epsilon(-)^{n(n-1)/2}(-\eta)^n \mathcal{C}\Gamma^{(n)}$ find which $n\bmod{4}$ give symmetric $\mathcal{C}\Gamma^{(n)}$.  The sum of $\binom{d}{n}$ must be $2^{\lfloor d/2\rfloor} (2^{\lfloor d/2\rfloor} + 1) / 2$.  This fixes $\epsilon, \eta$.  Odd $d$ require $\eta=(-1)^{d(d+1)/2}$ to preserve $\Gamma^{(d)}$.  Even~$d$ allow two choices of signs: consult the rows $d\pm 1$.
\vspace{-\baselineskip}
\begin{center}
\begin{tabular}[t]{rccc}\toprule
$d\bmod 8$ & $n$ & $\epsilon$ & $\eta$ \\\midrule\addlinespace[7pt]
\multirow{2}{*}[11pt]{\rlap{0$\langle$}}\multirow{2}{*}{2$\langle$}1   & 0, 1 & $-1$ & $-1$\\
\multirow{2}{*}{4$\langle$}3   & 1, 2 & $+1$ & $+1$\\
\multirow{2}{*}{6$\langle$}5   & 2, 3 & $+1$ & $-1$\\
\phantom{0$\langle$}7   & 0, 3 & $-1$ & $+1$\\\bottomrule
\end{tabular}
\end{center}

\paragraph{Reduced spinors.}
$M_{ab}\in so(s,t)$ acts as $\gamma_a \gamma_b$ on representations of the Clifford algebra.
But the $2^{\lceil d/2\rceil}$-dimensional representation is not irreducible as a representation of $so(s,t)$.

In even~$d$, Weyl (or chiral) spinors $\Gamma^{(d)}\lambda=\pm\lambda$ have $2^{d/2-1}$ real components.
Let $B$~be defined by $\Gamma_a^*=-\eta(-1)^t B\Gamma_a B^{-1}$.
Majorana spinors $\lambda^*=B\lambda$ exist for $s-t\equiv 0,\pm 1,\pm 2\bmod{8}$;
the case $s-t\equiv\pm 2$ requires $\eta=\mp (-1)^{d/2}$.
When $s-t\equiv 3,4,5$, a set of $2n$ spinors can be symplectic Majorana: $(\lambda^I)^*=B\Omega_{IJ}\lambda^J$ for $\Omega=((0,\mathbbm{1}_n);(-\mathbbm{1}_n,0))$.
(Symplectic) Majorana--Weyl spinors exist for $s-t\equiv 0,4\bmod{8}$.
The table also includes the real dimension of the minimal spinor.
\begin{center}
\vspace{-.5\baselineskip}
\begin{tabular}{c*{4}{>{ }l@{ }r<{ }}}\toprule
d &\multicolumn{2}{c}{$t\equiv 0$} &\multicolumn{2}{c}{$1$}
& \multicolumn{2}{c}{$2$} & \multicolumn{2}{c}{$3\bmod{4}$} \\ \midrule
1 & M     & 1 & M    & 1 &       &   & &   \\
2 & M$^-$ & 2 & MW   & 1 & M$^+$ & 2 & &   \\
3 & s     & 4 & M    & 2 & M     & 2 & s     & 4 \\
4 & sW    & 4 & M$^+$& 4 & MW    & 2 & M$^-$ & 4 \\
5 & s     & 8 & s    & 8 & M     & 4 & M     & 4 \\
6 & M$^+$ & 8 & sW   & 8 & M$^-$ & 8 & MW    & 4 \\
7 & M     & 8 & s    & 16 & s    & 16 & M    & 8 \\
8 & MW    & 8 & M$^-$& 16 & sW   & 16 & M$^+$& 16 \\
9 & M     & 16& M    & 16 & s    & 32 & s    & 32 \\
10& M$^-$ & 32& MW   & 16 & M$^+$& 32 & sW   & 32 \\
11& s     & 64& M    & 32 & M    & 32 & s    & 64 \\
12& sW    & 64& M$^+$& 64 & MW   & 32 & M$^-$ & 64\\\bottomrule
\end{tabular}
\end{center}

\paragraph{Flavour symmetries} of $N$ minimal spinors.
This is also the $R$-symmetry of the $N$-extended superalgebra.
For (symplectic) Majorana Weyl spinors, specify $N=(N_L,N_R)$ left/right-handed.
\begin{center}
\vspace{-.5\baselineskip}
\begin{tabular}{l@{}>{$}l<{$}}\toprule
M & \hspace{-2pt}\begin{cases} u(N)&\text{if $d$ even}\\so(N)&\text{if $d$ odd}\end{cases}\\
MW & : so(N_L)\times so(N_R) \\
s & : usp(N) \\
sW & : usp(N_L)\times usp(N_R)\\\bottomrule
\end{tabular}
\end{center}

\paragraph{Products of spinor representations.}
For odd $d=2m+1$, let $\mathcal{S}$ be a spinor representation of complex dimension $2^{m}$.
The symmetric product $S^2\mathcal{S}$ consists of $k$-forms with $k\equiv m\bmod{4}$.
Since $k$-forms and $(d-k)$-forms are the same representation, other descriptions can be given.
For the antisymmetric product $\Lambda^2\mathcal{S}$, take $k\equiv m-1\bmod{4}$.
See the list of forms in the table.
\begin{center}
\vspace{-.5\baselineskip}
\begin{tabular}{>{$}l<{$}*{6}{>{$}l<{$}}}\toprule
d & 1 & 3 & 5 & 7 & 9 & 11\\
\dim_{\mathbb{C}}\mathcal{S} & 1 & 2 & 4 & 8 & 16 & 32\\\midrule
S^2\mathcal{S} & 0 & 1 & 2 & 0,3 & 0,1,4 & 1,2,5\\
\Lambda^2\mathcal{S} & . & 0 & 0,1 & 1,2 & 2,3 & 0,3,4\\\bottomrule
\end{tabular}
\end{center}
For even $d=2m$, let $\mathcal{S}_{\pm}$ be the Weyl spinor representations of complex dimension $2^{m-1}$.
The tensor product $\mathcal{S}_{+}\otimes\mathcal{S}_{-}$ consists of $(m-1-2j)$-forms for $0\leq j\leq (m-1)/2$.
The symmetric products $S^2\mathcal{S}_{\pm}$ decompose into the (anti)-self-dual $m$-forms and $(m-4j)$-forms for $0<j\leq m/4$.
The antisymmetric products $\Lambda^2\mathcal{S}_{\pm}$ decompose into $(m-2-4j)$-forms for $0\leq j\leq (m-2)/4$.
\begin{center}
\vspace{-.5\baselineskip}
\begin{tabular}{>{$}l<{$}*{6}{>{$}l<{$}}}\toprule
d & 2 & 4 & 6 & 8 & 10 & 12\\
\dim_{\mathbb{C}}\mathcal{S}_{\pm} & 1 & 2 & 4 & 8 & 16 & 32\\\midrule
S^2\mathcal{S}_{\pm} & 1^{\dagger} & 2^{\dagger} & 3^{\dagger} & 0,4^{\dagger} & 1,5^{\dagger} & 2,6^{\dagger}\\
\Lambda^2\mathcal{S}_{\pm} & . & 0 & 1 & 2 & 3 & 0,4\\
\mathcal{S}_{+}\otimes\mathcal{S}_{-} & 0 & 1 & 0,2 & 1,3 & 0,2,4 & 1,3,5\\
\bottomrule
\end{tabular}
\vspace{-.5\baselineskip}
\end{center}
Note that
$\begin{aligned}[t]
S^2(\mathcal{S}_{+}\oplus\mathcal{S}_{-})&=S^2\mathcal{S}_{+}\oplus(\mathcal{S}_{+}\otimes\mathcal{S}_{-})\oplus S^2\mathcal{S}_{-}\\
\Lambda^2(\mathcal{S}_{+}\oplus\mathcal{S}_{-})&=\Lambda^2\mathcal{S}_{+}\oplus(\mathcal{S}_{+}\otimes\mathcal{S}_{-})\oplus \Lambda^2\mathcal{S}_{-}
\end{aligned}$

\section{Supersymmetry algebras}

\paragraph{The Poincar\'e algebra} is $\mathbb{R}^{s,t} \rtimes so(s,t)$, the semi-direct product of translations by rotations.
Namely, $[P_{a},P_{b}]=0$, $[M_{ab},P_{c}]=2ih_{c[a}P_{b]}$, and $[M_{ab},M^{cd}]=4ih_{[a}^{[c}M_{b]}^{d]}$.

\smallskip

\paragraph{Super-Poincar\'e algebra.}
Add supercharges in some spinor representation~$Q$ of the Poincar\'e algebra (so $[P_{a},Q]=0$).
Their anticommutator transforms in the representation $S^2 Q$ and should include the one-form~$P$.
Depending on $s,t$ they can include other $k$-forms~$Z$, called central charges because $[P,Z]=[Z,Z]=0$.
The super-Poincar\'e algebra is $((\mathbb{R}^{s,t}\times Z).Q)\rtimes (so(s,t)\times R)$, where the $R$-symmetry acts on~$Q$.
This Lie superalgebra is graded: $\operatorname{gr}(\mathbb{R}^{s,t}\times Z)=-2$, $\operatorname{gr}(Q)=-1$, and $\operatorname{gr}(so(s,t)\times R)=0$.
The supertranslations consist of $(\mathbb{R}^{s,t}\times Z).Q$.

\smallskip

\paragraph{Example: M-theory algebra.}  $d=10+1$ super-Poincar\'e algebra with $Q=\text{Majorana}$.
Since $S^2 Q$ has $1$, $2$, and $5$-forms, there are $2$-form and $5$-form central charges $Z_{(2)}$ and~$Z_{(5)}$
(under which M2 and M5 branes are charged):
\vspace{-.5\baselineskip}
\begin{align*}
\{Q_{\alpha},Q_{\beta}\}&=(\gamma^{M}C)_{\alpha\beta} P_{M}+\frac{1}{2}(\gamma_{MN}C)_{\alpha\beta} Z_{(2)}^{MN}\\[-.5\baselineskip]
& \quad + \frac{1}{5!}(\gamma_{MNPQR}C)_{\alpha\beta} Z_{(5)}^{MNPQR}
\end{align*}
\vspace{-1\baselineskip}

\noindent Altogether the M-theory algebra is $osp(1|32)$.

\smallskip

\paragraph{Superconformal algebras} are the same as super $AdS_{d+1}$.
The bosonic part is $so(d,2)$ and $R$-symmetries.
As a supermatrix: $\begin{pmatrix}so(d,2)& Q+S\\ Q-S&R\end{pmatrix}$ or $so(d,2)\leftrightarrow R$.  Note that $\{Q,S\}$ contains~$R$.
For $d=2$, the finite conformal algebra is $so(2,2)=so(2,1)\oplus so(2,1)$, sum of two $d=1$ algebras, so the superalgebra is sum of two $d=1$ superalgebras.
\begin{center}
\vspace{-.5\baselineskip}
\begin{tabular}{lllc}\toprule
$d$& Superalgebra& $R$-symmetries & \#Q+\#S\\ \midrule
$1$&  $osp(N|2)$ & $o(N)$    & $2N$ \\
   &  $su(N|1,1)$  &$su(N)\oplus u(1)$ for $N\neq 2$ &$ 4N$ \\
   &  $su(2|1,1)           $    &$su(2)              $ &$ 8   $\\
   &  $osp(4^*|2N)         $    &$su(2)\oplus usp(2N)$ &$ 8N  $\\
   &  $G(3)                $    &$G_2                $ &$ 14  $\\
   &  $F^0(4)                $    &$so(7)              $ &$ 16  $\\
   &  $D^0(2,1,\alpha)     $    &$su(2)\oplus su(2)  $ &$  8  $\\  \midrule
$3$&$ osp(N|4)   $ &$ so(N) $&$ 4N $\\   \midrule
$4$&$ su(2,2|N)  $ &$su(N)\oplus u(1)$ for $N\neq 4$&$ 8N$\\
   &$ su(2,2|4)  $ &$  su(4)$ & 32\\  \midrule
$5$&$ F^2(4)       $ &$ su(2) $ & 16 \\       \midrule
$6$&$ osp(8^*|N) $ &$  usp(N)\ \ (N $ even)& $8N$ \\
\bottomrule
\end{tabular}
\end{center}


\section{Supermultiplets with spins $\leq 1$}

\paragraph{For $16$ supercharges}, there is only the vector.

\paragraph{For $8$ supercharges}, vector and hyper.

\paragraph{For $4$ supercharges}, vector, chiral, linear multiplets.

\paragraph{For $2$ supercharges}, vector, chiral, linear, Fermi, \ldots{}

%\section{List of theories}
%
%\paragraph{Pure supergravities} in $4\leq d\leq 11$.  Gravity is topological in $d=3$.  The maximum number of supercharges $Q=32$ forbids $d>11$.  A priori, all $Q=4k$ are possible.  Focus on $32, 16, 8, 4$.
%\begin{center}\begin{tabular}{lcccc}\toprule
%$d$ &$Q=32$& $16$  &$8$  & $4$  \\ \midrule
%11& \checkmark & &&  \\
%10& $\stackrel{ IIB}{(2,0)}\ \stackrel{ IIA}{(1,1)}$ &
%$\stackrel{I}{(1,0)}$
%& & \\
%9 & \checkmark & \checkmark &&\\
%8 & \checkmark & \checkmark &&\\
%7 & \checkmark & \checkmark &&\\
%6 & $(2,2)$& $(2,0) \ (1,1)$ &
%$(1,0)$ &\\
%5 & \checkmark & \checkmark & \checkmark &\\
%4 &$N=8$ &$N=4$ & $N=2$
%&$N=1$\\\bottomrule
%\end{tabular}\end{center}

\bigskip
\vfill
\setlength{\parindent}{0pt}

%\begin{itemize}
%\item Say that only abelian $R$-symmetries can mix with global symmetries along the RG-flow.
%\end{itemize}


[1] \href{http://arxiv.org/abs/hep-th/9910030}{Tools for supersymmetry} by Antoine Van Proeyen

[2] Various Wikipedia articles.

\end{document}
