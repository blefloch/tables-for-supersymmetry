\documentclass[10pt,letterpaper]{article}
%\usepackage{savetrees}
\usepackage[margin=1.2cm]{geometry}
\setlength{\parskip}{0pt}
\setlength{\baselineskip}{10pt}
\usepackage{microtype}
\usepackage{amsfonts,amssymb,mathtools,array,booktabs,multirow,bbm,hyperref}
\usepackage{latexsym}
\usepackage{expl3}
\usepackage{etoolbox}
%%% Sorry the next two lines are a bit horrible
\renewcommand{\section}[1]{\stepcounter{section}\medskip\pagebreak[1]\noindent\textbf{\large\arabic{section}.\ #1}\smallskip}
\renewcommand{\paragraph}[1]{\par\noindent\textbf{#1}}
\newcommand{\definition}{\par\textbf{Def:} \ignorespaces}
\newcommand{\defined}[1]{\emph{#1}}

\newcommand{\myA}{}
\newcommand{\myB}{}
\newcommand{\ring}[1]{\mathbb{#1}}
\ExplSyntaxOn
\newcommand{\lie}[1]{
  \mathfrak {
    \str_case:nnF {#1}
      {
        { sl } { sl }
        { so } { so }
        { su } { su }
        { u } { u }
        { o } { o }
        { sp } { sp }
        { usp } { usp }
        { osp } { osp }
        { A } { a }
        { B } { b }
        { C } { c }
        { D } { d }
        { E } { e }
        { F } { f }
        { G } { g }
        { P } { p }
        { Q } { q }
        { UQ } { uq }
        { W } { w }
        { S } { s }
        { \tilde S } { \tilde s }
        { \tilde{S} } { \tilde{s} }
        { H } { h }
      } { \ERROR #1 }
    }
}
\ExplSyntaxOff
%\robustify\tilde

\newcommand{\undovskip}{\relax\ifvmode\unskip\fi}
\newenvironment{tab}{\center\undovskip\vspace{.1\baselineskip}\tabular}{\endtabular\endcenter\undovskip\vspace{.1\baselineskip}}

\begin{document}
\twocolumn\raggedbottom

\noindent\textbf{\Large Tables for supersymmetry.} Based on [1-2].

\noindent ``Ed.'': Bruno Le Floch, Princeton University, \today.

\section{Simple Lie (super)algebras}

\paragraph{Complex case.}
Infinite series $\lie{A}_{n\geq 1}$, $\lie{B}_{n\geq 1}$, $\lie{C}_{n\geq 1}$, $\lie{D}_{n\geq 2}$ with
$\lie{A}_1=\lie{B}_1=\lie{C}_1$, $\lie{B}_2=\lie{C}_2$, $\lie{D}_2=\lie{A}_1\oplus \lie{A}_1$, $\lie{D}_3=\lie{A}_3$.
Five exceptions with $\dim(\lie{E}_6)=78$, $\dim(\lie{E}_7)=133$, $\dim(\lie{E}_8)=248$, $\dim(\lie{F}_4)=52$, $\dim(\lie{G}_2)=14$.
\begin{tab}{*{3}{>{$}l<{$}}l}\toprule
  \text{Type} & \text{Dimension} & \text{Lie algebra} \\\midrule
  \lie{A}_n & n(n+2) & \lie{sl}(n+1,\ring{C}) =\{\text{traceless}\} \\
  \lie{B}_n & n(2n+1) & \lie{so}(2n+1,\ring{C}) =\{\text{antisymmetric}\} \\
  \lie{C}_n & n(2n+1) & \lie{sp}(2n,\ring{C})
	=\left\{\left(\begin{smallmatrix}0&\mathbbm{1}_n\\-\mathbbm{1}_n&0\end{smallmatrix}\right)\times\text{symmetric}\right\}
  \\
  \lie{D}_n & n(2n-1) & \lie{so}(2n,\ring{C}) =\{\text{antisymmetric}\} \\
  \bottomrule
\end{tab}

\paragraph{Real case.}  Let
$\lie{sl}(n)=\lie{sl}(n,\ring{R})$,
$\lie{sp}(2n)=\lie{sp}(2n,\ring{R})$,
$\lie{su}^*(2n)=\lie{sl}(n,\ring{H})$,
$\lie{so}^*(2n)=\lie{o}(n,\ring{H})$,
$\lie{usp}(2m,2n)=\lie{u}(m,n,\ring{H})$.
A Lie algebra is \defined{compact} if it exponentiates to a compact Lie group.
In $\lie{E}_{r(s)}$, $s$ is the number of $(\text{non-compact})-(\text{compact})$ generators.
\begin{tab}{@{}c@{ }lll}\toprule
& Real form & \hspace{-1em}Max compact subalgebra & Range \\
\midrule
\multirow{4}{*}{\rotatebox{90}{$sl(n)$}}
& $\lie{su}(n)$ & compact & \\
& $\lie{sl}(n)$ & $\lie{so}(n)$ & \\
& $\lie{su}(n-p,p)$ & $\lie{su}(n-p)\oplus \lie{su}(p)\oplus \lie{u}(1)$ & $0<p<n$ \\
& $\lie{su}^*(n)$ & $\lie{usp}(n)$ & $n$ even \\
\midrule
\multirow{3}{*}{\rotatebox{90}{$\lie{so}(n)$}}
& $\lie{so}(n)$& compact & \\
& $\lie{so}(p,n-p)$& $\lie{so}(p)\oplus \lie{so}(n-p)$ & $0<p<n$ \\
& $\lie{so}^*(2n)$   & $\lie{u}(n)$ & $n$ even \\
\midrule
\multirow{3}{*}{\rotatebox{90}{$\lie{sp}(2n)$}}
& $\lie{usp}(2n)$ & compact & \\
& $\lie{sp}(2n)$  & $\lie{u}(n)$ & \\
& $\lie{usp}(2n-2p,2p)$ & $\lie{usp}(2n-2p)\oplus \lie{usp}(2p)$ & $0<p<n$ \\
\midrule
\multicolumn{4}{c}{%
  \begin{tabular}[c]{ll}
  $\lie{E}_{6(-78)}$ & compact \\
  $\lie{E}_{6(-26)}$ & $F_4$ \\
  $\lie{E}_{6(-14)}$ & $\lie{so}(10)\oplus \lie{so}(2)$\\
  $\lie{E}_{6(2)}$ & $\lie{su}(6)\oplus \lie{su}(2)$\\
  $\lie{E}_{6(6)}$ & $\lie{usp}(8)$\\
  \midrule
  $\lie{E}_{7(-133)}$& compact \\
  $\lie{E}_{7(-25)}$& $\lie{E}_{6,-78}\oplus \lie{so}(2)$ \\
  $\lie{E}_{7(-5)}$& $\lie{so}(12)\oplus \lie{su}(2)$ \\
  $\lie{E}_{7(7)}$& $\lie{su}(8)$
  \end{tabular}\quad
  \begin{tabular}[c]{ll}
  $\lie{E}_{8(-248)}$& compact\\
  $\lie{E}_{8(-24)}$&$\lie{E}_{7,-133}\oplus \lie{su}(2)$\\
  $\lie{E}_{8(8)}$&$\lie{so}(16)$\\
  \midrule
  $\lie{G}_{2(-14)}$ & compact \\
  $\lie{G}_{2(2)}$ & $\lie{su}(2)\oplus \lie{su}(2)$ \\
  \midrule
  $\lie{F}_{4(-52)}$ & compact \\
  $\lie{F}_{4(-20)}$ & $\lie{so}(9)$ \\
  $\lie{F}_{4(4)}$ & $\lie{usp}(6)\oplus \lie{su}(2)$ \\
  \end{tabular}
}\\\bottomrule
\end{tab}

\paragraph{Accidental isomorphisms.}
\begin{center}
\vspace{-2.5\baselineskip}
\begin{minipage}[t]{.55\linewidth}
\begin{align*}
\lie{so}(2)&= \lie{u}(1), \quad \lie{so}(1,1)=\ring{R}\\
\lie{so}(3)&= \lie{su}(2)=\lie{su}^*(2)\\
\lie{so}(2,1) &=\lie{su}(1,1)=\lie{sl}(2)=\lie{sp}(2)\\
\lie{so}(4)&=\lie{su}(2)\oplus \lie{su}(2)\\
\lie{so}(3,1)&=\lie{sl}(2,\ring{C})=\lie{sp}(2,\ring{C})\\
\lie{so}(2,2)&=\lie{sl}(2)\oplus \lie{sl}(2)\\
\lie{so}^*(4)&=\lie{su}(1,1)\oplus \lie{su}(2)\\
\lie{so}(5)&=\lie{usp}(4)
\end{align*}
\end{minipage}%
\begin{minipage}[t]{.45\linewidth}
\begin{align*}
\lie{so}(4,1)&=\lie{usp}(2,2)\\
\lie{so}(3,2)&=\lie{sp}(4)\\
\lie{so}(6)&=\lie{su}(4)\\
\lie{so}(5,1)&=\lie{su}^*(4)\\
\lie{so}(4,2)&=\lie{su}(2,2)\\
\lie{so}(3,3)&=\lie{sl}(4)\\
\lie{so}^*(6)&=\lie{su}(3,1)\\
\lie{so}^*(8)&=\lie{so}(6,2)
\end{align*}
\end{minipage}
\end{center}

\paragraph{Classical Lie superalgebras:}
the bosonic algebra acts on the fermionic generators in a completely reducible representation.
This excludes Cartan-type superalgebras $\lie{W}(n)$, $\lie{S}(n)$, $\lie{\tilde S}(n)$ and $\lie{H}(n)$.
In this table, $m,n\geq 1$ and we do not list purely bosonic Lie algebras.
The factor $\ring{C}$ of $\lie{sl}(m|n)$ must be removed if $m=n$.
\begin{tab}{lll}\toprule
& Bosonic algebra & Fermionic repr. \\\midrule
$\lie{sl}(m|n)$ & $\lie{sl}(m,\ring{C})\oplus \lie{sl}(n,\ring{C})\oplus\ring{C}$ & $(m,\overline{n})\oplus(\overline{m},n)$ \\
$\lie{osp}(m|2n)$ & $\lie{so}(m,\ring{C}) \oplus \lie{sp}(2n)$ & $(m,2n)$ \\
$\lie{D}(2,1,\alpha)$ & $\lie{sl}(2,\ring{C})^3$ & $(2,2,2)$ \\
$\lie{F}(4)$ & $\lie{so}(7,\ring{C})\oplus \lie{sl}(2,\ring{C})$ & $(8,2)$ \\
$\lie{G}(3)$ & $\lie{G}_2\oplus \lie{sl}(2,\ring{C})$ & $(7,2)$ \\
$\lie{P}(m)$ & $\lie{sl}(m+1,\ring{C})$ & $\text{sym}\oplus(\text{antisym})^*$ \\
$\lie{Q}(m)$ & $\lie{sl}(m+1,\ring{C})$ & adjoint\\
\bottomrule
\end{tab}

\paragraph{Real forms of Lie superalgebras,}
starting from their compact form ($p=q=0$).  $\lie{P}(m)$ has no compact form.
Here, $m,n\geq 1$, $0\leq p\leq m/2$, $0\leq q\leq n/2$.
The forms $\lie{su}^*$, $\lie{osp}^*$, $\lie{Q}^*$ only exist for even rank; $\lie{sl}'$ only if $m=n$.
\begin{tab}{*{2}{>{$}l<{$}}}\toprule
\text{Real form} & \text{Bosonic algebra}  \\ \midrule
\lie{su}(m-p,p|n-q,q) & \lie{su}(m-p,p)\oplus \lie{su}(n-q,q)\oplus \lie{u}(1)^{\mathsection}\\
\lie{sl}(m|n) & \lie{sl}(m)\oplus \lie{sl}(n)\oplus \lie{so}(1,1)^{\mathsection} \\
\lie{sl}'(n|n) \quad\, (m=n)& \lie{sl}(n,\ring{C})\\
\lie{su}^*(m|n) \:\: (m,n \text{ even}) & \lie{su}^*(m)\oplus \lie{su}^*(n)\oplus \lie{so}(1,1)^{\mathsection}\\
\midrule
\lie{osp}(m-p,p|2n) & \lie{so}(m-p,p)\oplus \lie{sp}(2n) \\
\multicolumn{2}{l}{$\lie{osp}^*(m|2n-2q,2q)$ ($m$ even)\quad $\lie{so}^*(m)\oplus \lie{usp}(2n-2q,2q)$} \\
\midrule
\lie{D}^p(2,1,\alpha) \;^{\mathparagraph} & \lie{so}(4-p,p)\oplus \lie{sl}(2)\quad (p=0,1,2)\\
\midrule
\lie{F}^p(4) \text{ for $p=0,3$} & \lie{so}(7-p,p)\oplus \lie{sl}(2) \\
\lie{F}^p(4) \text{ for $p=1,2$} & \lie{so}(7-p,p)\oplus \lie{su}(2) \\
\midrule
\lie{G}_s(3) \text{ for $s=-14,2$} & \lie{G}_{2(s)}\oplus sl(2) \\
\midrule
\lie{P}(m) & \lie{sl}(m+1) \\
\midrule
\lie{UQ}(m-p,p) & \lie{su}(m+1-p,p) \\
\lie{Q}(m) & \lie{sl}(m+1) \\
\lie{Q}^*(m) \quad (m \text{ odd}) & \lie{su}^*(m+1) \\
\bottomrule
\end{tab}

$^{\mathsection}$
For $m=n$, $\lie{u}(1)$ and $\lie{so}(1,1)$ factors are absent.
Additionally, one can project down to a single bosonic factor.

$^{\mathparagraph}$
The three $\lie{sl}(2)$ bosonic factors of $\lie{D}(2,1,\alpha)$ appear with weights $1$, $\alpha$ and $-1-\alpha$ in fermion anticommutators.
For $\lie{D}^0$ and $\lie{D}^2$, $\alpha$ is real.  For $\lie{D}^1$, $\alpha=1+ia$ with $a$ real.

\smallskip

\paragraph{Some isomorphisms:}
$\lie{su}(1,1|1)=\lie{sl}(2|1)=\lie{osp}(2|2)$
and $\lie{su}(2|1)=\lie{osp}(2^*|2,0)$
and $\lie{D}^p(2,1,\alpha=1)=\lie{osp}(4-p,p|2)$.

\section{Spinors}

\paragraph{Clifford algebra.}  Let $h_{ab}$ be diagonal with $s$~`$+1$' and $t$~`$-1$', and $d=s+t$.  The Clifford algebra $\{\Gamma_a,\Gamma_b\}=2h_{ab}$ has real dimension~$2^d$ and is isomorphic to a matrix algebra $M_{2^{\#}}(\bullet)$ with
\begin{tab}{>{$}r<{$}*{8}{>{$}c<{$}}}\toprule
s-t \bmod{8} & 0 & 1 & 2 & 3 & 4 & 5 & 6 & 7 \\
\bullet \text{\quad is} & \ring{R}& \ring{R}\oplus\ring{R} & \ring{R} & \ring{C} & \ring{H} & \ring{H}\oplus\ring{H} & \ring{H} &\ring{C}\\\bottomrule
\end{tab}
%with $\bullet = [\ring{R}, \ring{R}\oplus\ring{R}, \ring{R}, \ring{C}, \ring{H}, \ring{H}\oplus\ring{H}, \ring{H}, \ring{C}][s-t\bmod{8}]$.


\paragraph{Charge conjugation.}  $(-\eta)\Gamma_a^T=\mathcal{C}\Gamma_a\mathcal{C}^{-1}$ are conjugate for $\eta=\pm 1$ because they obey the same algebra.  Get $\mathcal{C}^T=-\varepsilon \mathcal{C}$ with $\varepsilon=\pm 1$ by transposing twice.
Let $\Gamma^{(n)}=\Gamma_{a_1\ldots a_n}$.
Using $\left(\mathcal{C}\Gamma^{(n)}\right) ^T= -\epsilon(-)^{n(n-1)/2}(-\eta)^n \mathcal{C}\Gamma^{(n)}$ find which $n\bmod{4}$ give symmetric $\mathcal{C}\Gamma^{(n)}$.  The sum of $\binom{d}{n}$ must be $2^{\lfloor d/2\rfloor} (2^{\lfloor d/2\rfloor} + 1) / 2$.  This fixes $\epsilon, \eta$.  Odd $d$ require $\eta=(-1)^{d(d+1)/2}$ to preserve $\Gamma^{(d)}$.  Even~$d$ allow two choices of signs: consult the rows $d\pm 1$.
\begin{tab}{rccc}\toprule
$d\bmod 8$ & $n$ & $\epsilon$ & $\eta$ \\\midrule\addlinespace[7pt]
\multirow{2}{*}[11pt]{\rlap{0$\langle$}}\multirow{2}{*}{2$\langle$}1   & 0, 1 & $-1$ & $-1$\\
\multirow{2}{*}{4$\langle$}3   & 1, 2 & $+1$ & $+1$\\
\multirow{2}{*}{6$\langle$}5   & 2, 3 & $+1$ & $-1$\\
\phantom{0$\langle$}7   & 0, 3 & $-1$ & $+1$\\\bottomrule
\end{tab}

\paragraph{Reduced spinors.}
$M_{ab}\in \lie{so}(s,t)$ acts as $\gamma_a \gamma_b$ on representations of the Clifford algebra.
But the $2^{\lceil d/2\rceil}$-dimensional representation is not irreducible as a representation of $so(s,t)$.

In even~$d$, Weyl (or chiral) spinors $\Gamma^{(d)}\lambda=\pm\lambda$ have $2^{d/2-1}$ real components.
Let $B$~be defined by $\Gamma_a^*=-\eta(-1)^t B\Gamma_a B^{-1}$.
Majorana spinors $\lambda^*=B\lambda$ exist for $s-t\equiv 0,\pm 1,\pm 2\bmod{8}$;
the case $s-t\equiv\pm 2$ requires $\eta=\mp (-1)^{d/2}$.
When $s-t\equiv 3,4,5$, a set of $2n$ spinors can be symplectic Majorana: $(\lambda^I)^*=B\Omega_{IJ}\lambda^J$ for $\Omega=((0,\mathbbm{1}_n);(-\mathbbm{1}_n,0))$.
(Symplectic) Majorana--Weyl spinors exist for $s-t\equiv 0,4\bmod{8}$.
The table also includes the real dimension of the minimal spinor.
\begin{tab}{c*{4}{>{ }l@{ }r<{ }}}\toprule
d &\multicolumn{2}{c}{$t\equiv 0$} &\multicolumn{2}{c}{$1$}
& \multicolumn{2}{c}{$2$} & \multicolumn{2}{c}{$3\bmod{4}$} \\ \midrule
1 & M     & 1 & M    & 1 &       &   & &   \\
2 & M$^-$ & 2 & MW   & 1 & M$^+$ & 2 & &   \\
3 & s     & 4 & M    & 2 & M     & 2 & s     & 4 \\
4 & sW    & 4 & M$^+$& 4 & MW    & 2 & M$^-$ & 4 \\
5 & s     & 8 & s    & 8 & M     & 4 & M     & 4 \\
6 & M$^+$ & 8 & sW   & 8 & M$^-$ & 8 & MW    & 4 \\
7 & M     & 8 & s    & 16 & s    & 16 & M    & 8 \\
8 & MW    & 8 & M$^-$& 16 & sW   & 16 & M$^+$& 16 \\
9 & M     & 16& M    & 16 & s    & 32 & s    & 32 \\
10& M$^-$ & 32& MW   & 16 & M$^+$& 32 & sW   & 32 \\
11& s     & 64& M    & 32 & M    & 32 & s    & 64 \\
12& sW    & 64& M$^+$& 64 & MW   & 32 & M$^-$ & 64\\\bottomrule
\end{tab}

\paragraph{Flavour symmetries} of $N$ minimal spinors.
This is also the $R$-symmetry of the $N$-extended superalgebra.
For (symplectic) Majorana Weyl spinors, specify $N=(N_L,N_R)$ left/right-handed.
\begin{tab}{l@{}>{$}l<{$}}\toprule
M & \hspace{-2pt}\begin{cases} \lie{u}(N)&\text{if $d$ even}\\\lie{so}(N)&\text{if $d$ odd}\end{cases}\\
MW & : \lie{so}(N_L)\times \lie{so}(N_R) \\
s & : \lie{usp}(N) \\
sW & : \lie{usp}(N_L)\times \lie{usp}(N_R)\\\bottomrule
\end{tab}

\paragraph{Products of spinor representations.}
For odd $d=2m+1$, let $\mathcal{S}$ be a spinor representation of complex dimension $2^{m}$.
The symmetric product $S^2\mathcal{S}$ consists of $k$-forms with $k\equiv m\bmod{4}$.
Since $k$-forms and $(d-k)$-forms are the same representation, other descriptions can be given.
For the antisymmetric product $\Lambda^2\mathcal{S}$, take $k\equiv m-1\bmod{4}$.
See the list of forms in the table.
\begin{tab}{>{$}l<{$}*{6}{>{$}l<{$}}}\toprule
d & 1 & 3 & 5 & 7 & 9 & 11\\
\dim_{\ring{C}}\mathcal{S} & 1 & 2 & 4 & 8 & 16 & 32\\\midrule
S^2\mathcal{S} & 0 & 1 & 2 & 0,3 & 0,1,4 & 1,2,5\\
\Lambda^2\mathcal{S} & . & 0 & 0,1 & 1,2 & 2,3 & 0,3,4\\\bottomrule
\end{tab}
For even $d=2m$, let $\mathcal{S}_{\pm}$ be the Weyl spinor representations of complex dimension $2^{m-1}$.
The tensor product $\mathcal{S}_{+}\otimes\mathcal{S}_{-}$ consists of $(m-1-2j)$-forms for $0\leq j\leq (m-1)/2$.
The symmetric products $S^2\mathcal{S}_{\pm}$ decompose into the (anti)-self-dual $m$-forms and $(m-4j)$-forms for $0<j\leq m/4$.
The antisymmetric products $\Lambda^2\mathcal{S}_{\pm}$ decompose into $(m-2-4j)$-forms for $0\leq j\leq (m-2)/4$.
\begin{tab}{>{$}l<{$}*{6}{>{$}l<{$}}}\toprule
d & 2 & 4 & 6 & 8 & 10 & 12\\
\dim_{\ring{C}}\mathcal{S}_{\pm} & 1 & 2 & 4 & 8 & 16 & 32\\\midrule
S^2\mathcal{S}_{\pm} & 1^{\dagger} & 2^{\dagger} & 3^{\dagger} & 0,4^{\dagger} & 1,5^{\dagger} & 2,6^{\dagger}\\
\Lambda^2\mathcal{S}_{\pm} & . & 0 & 1 & 2 & 3 & 0,4\\
\mathcal{S}_{+}\otimes\mathcal{S}_{-} & 0 & 1 & 0,2 & 1,3 & 0,2,4 & 1,3,5\\
\bottomrule
\end{tab}
Note that
$\begin{aligned}[t]
S^2(\mathcal{S}_{+}\oplus\mathcal{S}_{-})&=S^2\mathcal{S}_{+}\oplus(\mathcal{S}_{+}\otimes\mathcal{S}_{-})\oplus S^2\mathcal{S}_{-}\\
\Lambda^2(\mathcal{S}_{+}\oplus\mathcal{S}_{-})&=\Lambda^2\mathcal{S}_{+}\oplus(\mathcal{S}_{+}\otimes\mathcal{S}_{-})\oplus \Lambda^2\mathcal{S}_{-}
\end{aligned}$

\section{Supersymmetry algebras}

\paragraph{The Poincar\'e algebra} is $\ring{R}^{s,t} \rtimes \lie{so}(s,t)$, the semi-direct product of translations by rotations.
Namely, $[P_{a},P_{b}]=0$, $[M_{ab},P_{c}]=2ih_{c[a}P_{b]}$, and $[M_{ab},M^{cd}]=4ih_{[a}^{[c}M_{b]}^{d]}$.

\smallskip

\paragraph{Super-Poincar\'e algebra.}
Add supercharges in some spinor representation~$Q$ of the Poincar\'e algebra (so $[P_{a},Q]=0$).
Their anticommutator transforms in the representation $S^2 Q$ and should include the one-form~$P$.
Depending on $s,t$ they can include other $k$-forms~$Z$, called central charges because $[P,Z]=[Z,Z]=0$.
The super-Poincar\'e algebra is $((\ring{R}^{s,t}\times Z).Q)\rtimes (\lie{so}(s,t)\times R)$, where the $R$-symmetry acts on~$Q$.
This Lie superalgebra is graded: $\operatorname{gr}(\ring{R}^{s,t}\times Z)=-2$, $\operatorname{gr}(Q)=-1$, and $\operatorname{gr}(\lie{so}(s,t)\times R)=0$.
The supertranslations consist of $(\ring{R}^{s,t}\times Z).Q$.

\smallskip

\paragraph{Example: M-theory algebra.}  $d=10+1$ super-Poincar\'e algebra with $Q=\text{Majorana}$.
Since $S^2 Q$ has $1$, $2$, and $5$-forms, there are $2$-form and $5$-form central charges $Z_{(2)}$ and~$Z_{(5)}$
(under which M2 and M5 branes are charged):
\vspace{-.5\baselineskip}
\begin{align*}
\{Q_{\alpha},Q_{\beta}\}&=(\gamma^{M}C)_{\alpha\beta} P_{M}+\frac{1}{2}(\gamma_{MN}C)_{\alpha\beta} Z_{(2)}^{MN}\\[-.5\baselineskip]
& \quad + \frac{1}{5!}(\gamma_{MNPQR}C)_{\alpha\beta} Z_{(5)}^{MNPQR}
\end{align*}
\vspace{-1\baselineskip}

\noindent Altogether the M-theory algebra is $\lie{osp}(1|32)$.

\smallskip

\paragraph{Superconformal algebras} are the same as super $AdS_{d+1}$.
The bosonic part is $\lie{so}(d,2)$ and $R$-symmetries.
As a supermatrix: $\begin{pmatrix}\lie{so}(d,2)& Q+S\\ Q-S&R\end{pmatrix}$ or $\lie{so}(d,2)\leftrightarrow R$.  Note that $\{Q,S\}$ contains~$R$.
For $d=2$, the finite conformal algebra is $\lie{so}(2,2)=\lie{so}(2,1)\oplus \lie{so}(2,1)$, sum of two $d=1$ algebras, so the superalgebra is sum of two $d=1$ superalgebras.
\begin{tab}{lllc}\toprule
$d$& Superalgebra& $R$-symmetries & \#Q+\#S\\ \midrule
$1$&  $\lie{osp}(N|2)$ & $\lie{o}(N)$    & $2N$ \\
   &  $\lie{su}(N|1,1)$  &$\lie{su}(N)\oplus \lie{u}(1)$ for $N\neq 2$ &$ 4N$ \\
   &  $\lie{su}(2|1,1)           $    &$\lie{su}(2)              $ &$ 8   $\\
   &  $\lie{osp}(4^*|2N)         $    &$\lie{su}(2)\oplus \lie{usp}(2N)$ &$ 8N  $\\
   &  $\lie{G}(3)                $    &$\lie{G}_2                $ &$ 14  $\\
   &  $\lie{F}^0(4)                $    &$\lie{so}(7)              $ &$ 16  $\\
   &  $\lie{D}^0(2,1,\alpha)     $    &$\lie{su}(2)\oplus \lie{su}(2)  $ &$  8  $\\  \midrule
$3$&$ \lie{osp}(N|4)   $ &$ \lie{so}(N) $&$ 4N $\\   \midrule
$4$&$ \lie{su}(2,2|N)  $ &$\lie{su}(N)\oplus \lie{u}(1)$ for $N\neq 4$&$ 8N$\\
   &$ \lie{su}(2,2|4)  $ &$  \lie{su}(4)$ & 32\\  \midrule
$5$&$ \lie{F}^2(4)       $ &$ \lie{su}(2) $ & 16 \\       \midrule
$6$&$ \lie{osp}(8^*|N) $ &$  \lie{usp}(N)\ \ (N $ even)& $8N$ \\
\bottomrule
\end{tab}


\section{Supermultiplets with spins $\leq 1$}

\paragraph{For $16$ supercharges}, there is only the vector.

\paragraph{For $8$ supercharges}, vector and hyper.

\paragraph{For $4$ supercharges}, vector, chiral, linear multiplets.

\paragraph{For $2$ supercharges}, vector, chiral, linear, Fermi, \ldots{}

%\section{List of theories}
%
%\paragraph{Pure supergravities} in $4\leq d\leq 11$.  Gravity is topological in $d=3$.  The maximum number of supercharges $Q=32$ forbids $d>11$.  A priori, all $Q=4k$ are possible.  Focus on $32, 16, 8, 4$.
%\begin{center}\begin{tabular}{lcccc}\toprule
%$d$ &$Q=32$& $16$  &$8$  & $4$  \\ \midrule
%11& \checkmark & &&  \\
%10& $\stackrel{ IIB}{(2,0)}\ \stackrel{ IIA}{(1,1)}$ &
%$\stackrel{I}{(1,0)}$
%& & \\
%9 & \checkmark & \checkmark &&\\
%8 & \checkmark & \checkmark &&\\
%7 & \checkmark & \checkmark &&\\
%6 & $(2,2)$& $(2,0) \ (1,1)$ &
%$(1,0)$ &\\
%5 & \checkmark & \checkmark & \checkmark &\\
%4 &$N=8$ &$N=4$ & $N=2$
%&$N=1$\\\bottomrule
%\end{tabular}\end{center}


% \section{Types of manifolds}

% \definition A \defined{$G$-structure} on an n-dimensional real
% manifold~$X$: a $G$-subbundle of the $GL(n,\ring{R})$-principal bundle
% of tangent frames.

% \definition A manifold with
%  $O(\dim_{\ring{R}}X)$ structure \defined{Riemannian} if it has  is a manifold with an
% .  Any manifold admits such a structure
% because $O(n)$ is a deformation retract of $GL(n)$.

% \definition An \defined{almost complex manifold}

\bigskip
\vfill
\setlength{\parindent}{0pt}

%\begin{itemize}
%\item Say that only abelian $R$-symmetries can mix with global symmetries along the RG-flow.
%\end{itemize}


[1] \href{http://arxiv.org/abs/hep-th/9910030}{Tools for supersymmetry} by Antoine Van Proeyen

[2] Various Wikipedia articles.

\end{document}
